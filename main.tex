\documentclass[12pt] {article}
\usepackage[utf8]{inputenc}
\usepackage{booktabs}
\usepackage[flushleft]{threeparttable}
\usepackage[format=plain, justification=raggedright,labelsep=none,singlelinecheck=false,labelsep=newline,textfont={bf, it},labelfont={bf},font={stretch=2}]{caption}
\usepackage{hyperref}
\usepackage{setspace}
\usepackage{multirow,makecell}
\doublespacing
\title{apa tables}

\author{Dan (Overleaf Support) MacKinnon}
\date{August 2020}

\begin{document}
\maketitle

%\section{Introduction}

This example is adapted from \url{https://owl.purdue.edu/owl/research_and_citation/apa_style/apa_formatting_and_style_guide/apa_tables_and_figures.html}


\begin{table}[!htb]
    \begin{threeparttable}[b]
   
    \caption{Title}
    \label{tbl:1}
    \centering
    \begin{tabular}{lcccc}
        \makecell[c]{Stub \\ Heading} & \multicolumn{2}{c}{Column Spanner} & \multicolumn{2}{c}{Column Spanner}\\
        %\addlinespace[10pt]
        \cmidrule(lr){2-5} \\
        {} &  \makecell[c]{Column \\ Heading} &  \makecell[c]{Column \\ Heading} &  \makecell[c]{Column \\ Heading}  & \makecell[c]{Column \\ Heading} \\
        %\addlinespace[10pt]
        \cmidrule(lr){1-5}
        Row 1  & 123  & 234\tnote{a} & 456  & 789 \\
        Row 2  & 123 & 987 & 543 & 867 \\
         \cmidrule(lr){1-5}
        Row 3  & 123  & 234 & 456  & 789 \\
        Row 4  & 123 & 987 & 543 & 867 \\
       % \bottomrule
    \end{tabular}
    \vspace{-10pt}
    \begin{tablenotes}    
    \item  \doublespacing Note. \textnormal{This is a general note, referring to information about the entire table.}
    \item[a] Specific notes appear in a new paragraph.
    \end{tablenotes}    
    \end{threeparttable}
\end{table}

\end{document}
